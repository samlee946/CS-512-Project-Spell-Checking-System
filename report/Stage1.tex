
\begin{itemize} 
\item{The general system description: } 
The user just needs to type their text in the text area, click the detect button, then several correction suggestions will be revealed on the right. The user is also able to report their mistakes corresponding to the correct word.
\item{The three types of users (grouped by their data access/update rights): }
\begin{enumerate}
	\item {\textbf{Administrator}: Administrators are able to manage database, manage user privileges and roles, and check system logs.}
	\item {\textbf{Normal users}: The basic users of this system. They use the system to check whether their words are correct.}
	\item {\textbf{System maintenance personnel}: System maintenance personnel is responsible for the maintenance of the dictionary.}
\end{enumerate}

\item{The user's interaction modes: }
A user typically uses a keyboard to input text, and uses a mouse and a monitor to interact with this GUI spell checking system. 
\item{The real world scenarios: }
Please insert the real world scenarios in here, as follows. 
	\begin{itemize} 
	\item{Scenario1 description: }
	A general user, like a student who is learning English, and he wants to know if the words that he writes is correct.
	\item{System Data Input for Scenario1: }
	Words or sentences
	\item{Input Data Types for Scenario1: }
	String
	\item{System Data Output for Scenario1: }
	Recommendations for modification of wrong words
	\item{Output Data Types for Scenario1: }
	String
	\item{Scenario2 description: }
	An author who wants to check if all words he used are correct 
	\item{System Data Input for Scenario2: }
	Sentences 
	\item{Input Data Types for Scenario2: }
	String
	\item{System Data Output for Scenario2: }
	Sentence with highlights of wrong words
	\item{Output Data Types for Scenario2: }
	String
	\item{Scenario3 description: }
	A system maintenance person who wants to add some new words
	\item{System Data Input for Scenario3: }
	Words  
	\item{Input Data Types for Scenario3: }
	String
	\item{System Data Output for Scenario3: }
	A message of whether the operation was successful or not
	\item{Output Data Types for Scenario3: }
	String
	\item{Scenario4 description: }
	A system maintenance person who wants to delete some words or correct some wrong words
	\item{System Data Input for Scenario4: }
	Words  
	\item{Input Data Types for Scenario4: }
	String
	\item{System Data Output for Scenario4: }
	A message of whether the operation was successful or not
	\item{Output Data Types for Scenario4: }
	String
	\item{Scenario5 description: }
	An administrator wants to add a new system maintenance person
	\item{System Data Input for Scenario5: }
	Username and password  
	\item{Input Data Types for Scenario5: }
	String
	\item{System Data Output for Scenario5: }
	A message of whether the operation was successful or not
	\item{Output Data Types for Scenario5: }
	String
	\item{Scenario6 description: }
	An administrator wants to check the log of the system
	\item{System Data Input for Scenario6: }
	Username and password
	\item{Input Data Types for Scenario6: }
	String
	\item{System Data Output for Scenario6: }
	The log of the system
	\item{Output Data Types for Scenario6: }
	String
	\end{itemize}
Please repeat that pattern for each user type.
\item{Project Time line and Divison of Labor.}
Please insert here the time line and the corresponding implementation tasks.
\end{itemize}

